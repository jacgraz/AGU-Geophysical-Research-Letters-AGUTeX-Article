% LaTeX rebuttal letter example. 
% 
% Copyright 2019 Friedemann Zenke, fzenke.net
%
% Based on examples by Dirk Eddelbuettel, Fran and others from 
% https://tex.stackexchange.com/questions/2317/latex-style-or-macro-for-detailed-response-to-referee-report
% 
% Licensed under cc by-sa 3.0 with attribution required.
% See https://creativecommons.org/licenses/by-sa/3.0/
% and https://stackoverflow.blog/2009/06/25/attribution-required/

\documentclass[12pt]{article}
\usepackage[utf8]{inputenc}
\usepackage{lipsum} % to generate some filler text
\usepackage{fullpage}
\usepackage{xcolor}
\usepackage{float}
\usepackage{soul}
\usepackage{graphicx}
\usepackage{breakcites}
\usepackage{url}

% import Eq and Section references from the main manuscript where needed
% \usepackage{xr}
% \externaldocument{manuscript}

% package needed for optional arguments
\usepackage{xifthen}
\newcommand*{\blue}{\textcolor{blue}}


\begin{document}



\section*{Response to the reviewers of 2024GL113042 }

% General intro text goes here
We thank the reviewers for their critical assessment of our work, their positive feedback and very useful suggestions. 
In the following we address their concerns point by point.\\

\noindent

\begin{table}[H]
\centering
\begin{tabular}{ |c|c| } 
 \hline
 {\itshape italic text  } & {\itshape Comments of the reviewers}  \\ 
 roman text               & Our answers or comments   \\ 
 \blue{blue text}         & \blue{Verbatim quotes from the manuscript}  \\ 
 \hline
\end{tabular}
\end{table}

\noindent
A manuscript is included with highlighted differences and modifications applied during revision is submitted as well. 


\subsection*{Editor Comment}
\textit{Thank you for submitting "Observation of falling snowflakes orientation at sheltered and unsheltered sites" [Paper \#2024GL113042] to Geophysical Research Letters. I have received 2 reviews of your manuscript, which are included below and/or attached. As you can see, the reviews indicate that major revisions are needed before we can consider proceeding with your paper. I am therefore returning the paper to you so that you can make the necessary changes.
}\\

\noindent
In this document we address point-by-point the remarks of the two reviewers and make the necessary changes in the manuscript. 

\newpage
\subsection*{Reviewer 1}

\textit{The manuscript is well-written. The authors did a great job addressing the observational issues with the Multi-Angel Snowflake Camera (MASC). I particularly appreciate the reverse-engineering numerical experiments: idealized and realistic objects were generated to evaluate various methods for describing the orientation of those objects. The findings from the numerical experiment, together with the analysis of the in-situ data, will be a valuable addition to the literature and a useful resource for the remote sensing community. I recommend the manuscript be published in GRL, after the following issues are addressed:}\\

\noindent
We thank the Reviewer for their positive feedback and for the suggestions and comments listed below. 

\subsubsection*{Major comments:}
\begin{enumerate}
    \item \textit{The discussion about the impact of the environmental winds on the orientation of the snowflakes is not sound. Wind speed alone may not be sufficient to capture this impact. The distribution of wind components likely plays a more significant role.}

    We agree with the reviewer that the distribution of wind components and turbulence at the relevant scales is probably more important than simple wind speed that we could investigate in this work (as all the MASC measurements in the database have co-located wind speed information available).
    
    We carefully revised the manuscript to make sure not to overstate the role of wind speed alone and also to clarify when we cannot separate between the "natural" contribution of wind on the distribution of orientations and the "artificial" contribution due to the turbulence generated by strong wind interacting with the hollow measurement system of the MASC. A few changes are highlighted in the manuscript. We may cite for example in Sec.~3:

    \blue{Atmospheric conditions are expected to play a role in shaping orientation distributions: atmospheric turbulence is probably a key driver, as well as horizontal winds interacting with the instrument frame~\cite{Fitch_AMT_2021}. Turbulence has been shown to heavily impact the falling of particles in fluids in general~\cite{Petersen_JFM_2019,Brandt_ARFM_2022} and of frozen hydrometeors in particular~\cite{Nemes_JFM_2017,Li_JFM_2021}.}

    and later in the same section:

    \blue{This interpretation cannot unfortunately tell us any relevant information about the role of environmental turbulence, but only the likely effect of local turbulence introduced by the instrument.  Further research is therefore necessary to clarify these dynamics, in particular the role of natural turbulence in broadening the orientation distribution.}    
\end{enumerate}

\subsubsection*{Minor comments:}
\begin{enumerate}
    \item \textit{Including results on graupel in a study focused specifically on snowflakes raises questions about the sampling method.}

    Our intention was not to focus only on snowflakes (meaning aggregate snowflakes) but on all the hydrometeors that can be reasonably observed with a MASC instrument. We realized that we were not particularly careful in the naming, and we revised the manuscript with care preferring expressions as "snow", "snow particles", "snow hydrometeors", "ice-phase hydrometeors" instead of "snowflakes". The title itself is rephrased as: \blue{Observation of the orientation  of snow hydrometeors at sheltered and unsheltered sites}.

    We are however aware that retrieving orientation from MASC images of very flat hydrometeors (plates, dendrites) is likely less accurate than retrieving it from aggregate snowflakes or graupel. This happens because planar crystals become recognizable and classifiable with the method of~\cite{Praz_AMT_2017}, used in this paper, only when their basal facet face reasonably the cameras. We commented about this in the manuscript as well:
    
    \blue{The results, especially for planar crystal types, must be considered with extreme care. In fact, the classification method of~\cite{Praz_AMT_2017}, which is trained on a supervised set, is able to identify planar crystals when their basal facet is facing towards the MASC cameras, making them recognizable.  This introduces a bias as some orientations would not allow for classification. For example, a thin plate falling perfectly horizontal, would appear in MASC images as a thin column. }
\end{enumerate}

\clearpage
\subsection*{Reviewer 2}
\textit{The authors present an interesting analysis of the orientation behavior of ice crystals and snowflakes during falling. The analysis is based on a large database of ground-based in-situ observations (Multi-Angle Snowflake Camera). Dependencies of the orientation on observational setups (wind mitigation), environmental parameters and particle properties are discussed. A simulation experiment further reveals important impacts of using different methods to estimate the average orientation angle from multiple cameras.}\\


\textit{Overall, I find the paper very well written and also the non-expert reader can easily follow. I have only a number of minor comments and suggestions. One might of course question whether the topic is of interest to a wider community, as it is a rather specific topic. However, I see the topic itself as highly relevant for active and passive microwave remote sensing but also for estimation of ice particle sedimentation velocity. Therefore, I recommend the manuscript for publication in GRL.}\\

\noindent
We thank the reviewer for their positive feedback and for the many insightful suggestions. We tried to implement most of their comments in the revised version of the manuscript, and we addressed the specific comments here below. 

\subsubsection*{Comments:}

\begin{enumerate}
    
    \item \textit{There is a very recent laboratory study about ice particle orientation which should be added to reference list and I also highly suggest to discuss their results. They appear to be slightly contradictory to your study:} \cite{Stout_ACP_2024} \\

    We thank the reviewer for this relevant, recommendation.  This high quality study, with laboratory experiments using snowflake replicas in a liquid solution shows that planar crystals may be falling with regimes that do not have a preferential orientation around 0$^\circ$. We believe that, given some of the assumptions of~\cite{Stout_ACP_2024}, their results are somehow complementary to our work and point out to the need to better understand the role of turbulence in orientation of snow particles. Regarding the cited study, we have the following comments:
    \begin{itemize}
        \item A relevant limitation of~\cite{Stout_ACP_2024} work is that, quoting, "it  only considers quiescent conditions, as the authors want to know under what conditions particle instability still occurs, even without the addition of turbulence.". This is quite different with respect to our field-based experiments.
        \item The study focuses on planar crystals and it uses 12 sub-types of this  hydrometeors (from plates to dendrites). In our case, the planar crystals are all classified together into an individual class, which is also one of the most difficult to classify with the MASC and to observe (given the 35 microns resolution of the MASC itself). In this sense our study is coarser and cannot go to this level of detail, which could be a reason why we do not observe such behavior. 
        \item Interestingly, another recent study~\cite{Koebschall_EF_2023} using replicas of snow aggregates and a liquid solution, showed instead that the particles tend to orient such that the area projected in the direction of flow is maximized, which is the type of behavior that is consistent with our observations. This is very interesting as two separate indoor studies, with similar methods, seem to indicate that individual crystals of planar type and aggregate snowflakes may be behaving differently regarding preferential orientation.
        \item It is unfortunate that the study reports, as possible backup evidence and as driving motivation, that previous studies using MASC also showed off-zero preferential orientations. This is exactly what we showed, in our paper, to be due to the bias of averaging \textbf{absolute value} of orientation estimates taken from different viewpoints and it was one of the driving motivation of our publication.  
        \item A further limitation is that the study considers objects falling in a liquid, which implies a far lower density ratio compared to hydrometeors in air, which in turn can impact the settling dynamics as discussed in recent studies by one of our coauthors (see~\cite{Tinklenberg_JFM_2023,Tinklenberg_JFM_2024}).
    \end{itemize}

    We now modified the manuscript to give some space to commenting and elaborating on the findings of~\cite{Stout_ACP_2024}. At first, we include this study in Table~1. Then, in the Introduction, we mention this study as:\\
    
    \blue{ A recent laboratory study using snow crystals replicas falling in an alcoholic solution~\cite{Stout_ACP_2024}, however, shows that in quiescent conditions individual planar crystals may have a spiralling fall regime with preferential orientations that depart from 0$^\circ$. A study with similar methods, but focused on larger snow aggregates, instead showed that horizontal orientation is preferred for these hydrometeors~\cite{Koebschall_EF_2023}. It is to be remarked that those and other studies, in considering objects falling in liquids, use a solid-to-fluid density ratio which is orders lower than in atmospheric precipitation. This may have far-reaching consequences in the settling dynamics~\cite{Tinklenberg_JFM_2023,Tinklenberg_JFM_2024}.  }\\

    And in the Conclusions:\\
    \blue{It must be noted that, at least in quiescent conditions and for some types of planar crystals, spiralling falling modes around a non-zero angle have been observed in the laboratory~\cite{Stout_ACP_2024} but it remains to be seen if this behavior can be observed and isolated also in natural conditions. With this respect, our database is not refined enough to be able to isolate efficiently individual types of planar crystals. }\\
    
    \item \textit{L. 37: Not sure if "standard deviation" is a term that is suited for the plain language summary. Maybe one can formulate simpler: "typical variability of the orientation angle\dots"}\\

    \noindent
    We agree with the reviewer, and we implemented their suggestion. \\

    \item \textit{Introduction: Honestly, comparing it with the other sections, I find the introduction a bit weak. Considering the wider audience addressed by GRL, I think the importance of ice particle orientation should be emphasized and explained more. It is certainly true that most radar polarimetric quantities are sensitive to orientation (L. 50-52). But this is by far not all in my opinion. Particle orientation also impacts the scattering signal in passive microwave (MW) observations (both from space and from ground). Many space-borne MW sensors are also polarization-sensitive. Considering the enormous efforts undertaken for example by ECMWF and other services to assimilate those all-sky MW observations, this aspect should be mentioned and cited. Another aspect is that an ice particle falling with a preferred orientation will have a very different sedimentation velocity. So characterizing the particle orientation is quite key to simulate its terminal velocity (which is quite an important parameter in any cloud model). Ice particle orientation is also relevant for determining cloud radiative properties such as optical depth or albedo.}
    
    We thank the reviewer for the suggestion. While we left the second part of the Introduction a bit more "specialized" (targeting an audience interested in snow imagers and their data processing), we realized that we did not provide a satasfactory broad overview, neglecting all the implications listed by the reviewer. We rephrased therefore the first part of the Introduction, removed a few technical paragraphs and included most of the suggestions of the reviewer. Some new paragraphs now read:
    
    \blue{The preferential orientation of falling hydrometeors is exploited by dual-polarization weather radars to retrieve information about the microphysical processes occurring in precipitation and to identify which type of hydrometeors are present and where. This is relevant for the interpretation of ground-based polarimetric radar data as well as for exploiting satellite-based observations and their (all-sky) assimilation, as done for example by the European Centre for Medium-Range Weather Forecasts ECMWF~\cite{Eyre_QJRMS_2022}. For example, recent studies showed that considering hydrometeor orientation in the assimilation of high-frequency radiometry may improve forecast in areas of cloud and precipitation~\cite{Barlakas_AMT_2021} and produce significant improvements in path-integrated quantities such as the Ice Water Path~\cite{Kaur_RS_2022}. }

    \blue{The orientation and falling regime also affects the settling velocity of ice phase hydrometers, a key parameter linking microphysical properties to other tangible quantities as the precipitation intensity, and that remains to be better understood and documented~\cite{Heymsfield_JAS_2004}.}\\

    \item \textit{L. 46: It would be good to add a reference for this statement. I question a bit whether this general statement is actually true. I completely agree for plate-like crystals or large aggregates. But what is for example about bullet rosettes in high cirrus clouds? Or columns/needles? Do we have solid knowledge about their falling orientation? Also the references you provide a few sentences later (\cite{Noel_JAMC_2005,Tinklenberg_JFM_2023}) focus only on planar crystals.}

    We believe the reviewer is right. We were biased tp think mostly in terms of larger snow particles (snow aggregates, graupel) or planar crystals, so our formulation was overstating the current knowledge. We rephrased fully this sentence as:\\
    
    \blue{Ice-phase hydrometeors are typically assumed to fall with a preferred orientation. Falling snow particles populating frozen clouds are generally considered to align preferentially in a way that maximizes the area projected in the vertical direction, as presented in studies conducting remote sensing retrievals and laboratory experiments.  Such experiments are documented for planar crystals~\cite{Noel_JAMC_2005,Matrosov_JAS_2005,Tinklenberg_JFM_2024} as well as for more complex and larger snowflakes and rimed hydrometeors~\cite{Kennedy_JAMC_2011, Ryzhkov_JAMC_2011, Koebschall_EF_2023}, while less is documented for other particle types such as needles, columns or rosettes. }\\


    \item \textit{L. 47: What do you mean by "relatively simple"? Personally, I find all those ice particles terribly complicated.}\\
    
    We agree that this sentence needed to be rephrased (see previous point).
    
    \item \textit{Table 1: I like the idea of having a table summarizing previous studies. But are there no studies that looked at columns or needles?}\\

    This is a good point. We did not find studies providing explicitly quantitative values of orientation distributions for columns and needles (which would be a very interesting topic to tackle). There are a few important work that deal with the orientation of columns/needles, such as~\cite{Klett_JAS_1995, Reinking_JAMC_1997, Melnikov_JAOT_2013}, They however investigate the theoretical effect of turbulence on orientation (Klett) or the effect of varying orientation on remote sensing observations (Reinking, Melnikov) rather than exploring or making hypothesis about the "nature" of orientation of columns and needles. 

    \item \textit{L. 140: Parenthesis around the reference Leinonen et al., 2021 should be removed.}
    
    \noindent
    It is now corrected.\\

    \item \textit{L. 157: Why "approximately"?}

    \noindent
    The reviewer is right. The sentence now reads:\\
    \blue{With \textit{Mean\_Abs} the final estimate would be 10$^\circ$, while for \textit{Mean} it would be 3.33$^\circ$}\\


    \item \textit{L. 211-215: One general thought regarding the fact that radar studies most often "retrieve" smaller range of orientation angles: If one looks at vertically pointing radar images, one always finds the lowest few hundred meters much more turbulent than the ice and snow cloud aloft. Surface friction is certainly responsible for this effect. So I am wondering if the often observable low turbulence in many ice clouds (for example indicated by small spectrum width) could be the reason why the orientation angle distribution is more narrow? Of course, the discrepancy might also be connected to unrealistic scattering properties often assumed by polarimetric shape retrievals.}

    This would be indeed a good research question. In order to answer to it, ideally it would be needed to collect MASC data (or data with a similar imager) also in the snow clouds aloft the surface. There are significant advances in this technology (imagers carried by tethered balloons for example) but we are not there yet. Airborne imagers (on research aircrafts), for which many data exist, are not able to preserve the natural orientation of the particles in their measurements. To add to the complexity, radar data themselves are often perturbed by ground clutter, they need a few hundred meters to form their beam or have internal hardware limitations (like TR-limiters) that affect short range measurements. Intuitively we can expect that, with respect to "quieter" high level clouds, near the surface we expect a broader variability driven by turbulence generated by surface friction.  

    \item \textit{Connected to this topic but even more complicated to answer is why some polarimetric radar observations seem to indicate a certain non-zero distribution of the particle orientation: \cite{Melnikov_JAOT_2013}}

    The work of~\cite{Melnikov_JAOT_2013} presents some retrievals of "fluttering intensity" from polarimetric radar data, being this intensity the standard deviation of the  distribution of orientation angles of ice-phase hydrometeors (which they assume  0-centered in their work, with zero being horizontal, although using the interval [$0,\pi$] in their definition). A possible source of confusion is that they mention sentences as "the median $\sigma_{\Theta}$ lies in an interval from 9 to 15\dots", but it is indeed the median of the standard deviations and not the median of the orientations. This study thus agrees with the other ones presented in our Table~1 and with the "standard" assumption of preferential horizontal orientation, at least in statistical sense. We decided to included this work in Table~1 and we thank the reviewer for the suggestion.    
    
\end{enumerate}

\bibliographystyle{apalike}
\bibliography{./references.bib}

\end{document}